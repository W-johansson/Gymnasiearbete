\section{Resultat}

Det mest lämpade neuronnätet blev det med följande parametrar:
\SI{5}{\styck} neuroner i det dolda lagret, en feedback delay på \SI{4}{\styck} värden,
inlärningshastighet på \num{0.3} och \SI{10}{\styck} element i minibatchen.
Dess kvadratiska förlusten var \num[round-precision=3]{0.002055976870427035}
vilket ger en genomsnittlig avvikelse på
\SI[round-precision=3]{2.7205728686321}{\percent}.

Lutningen från lutningsgivaren subtraherat med accelerationen dividerat med tyngdaccelerationen
blev inte lika med den beräknande lutningen från GPS:en, se figur~\ref{fig:phiresult},
där $\phi$ är lutningen från lutningsgivaren,
$\frac{a}{g}$ är accelerationen dividerat med tyngdaccelerationen
och $y$ är den beräknade lutningen från GPS:en.

% Resultat från lutningsgivare
\begin{sidewaysfigure}
	\centering
	\begin{tikzpicture}
		\begin{groupplot}[
				width = \textwidth, height = \textheight / 2,
				no markers,
				x unit = s, xlabel = Tid,
				y unit = \%, ylabel = Lutning,
				enlarge x limits = false,
				enlarge y limits = {abs = 0.1cm},
				group style = {
					group size = 1 by 2,
					vertical sep = 0pt,
					x descriptions at = edge bottom,
				},
				% Add a zero line
				extra y ticks=0, extra y tick labels=, extra y tick style={grid=major},
				extra x tick style = {grid = major},
				extra x tick labels=,
				extra x ticks= {59.94, 102.18, 122.8, 154.74, 214.72, 228.9},
				]
				\pgfplotstableread{phi.dat}\phidata;
				\nextgroupplot
				\addplot table[x=Time, y=phi] from \phidata;
				\addlegendentry{$\phi$}
				\addplot[dashed, color = DarkRed] table[x=Time, y expr = 100 * \thisrow{a} / 9.81] from \phidata;
				\addlegendentry{$\frac{a}{g}$}
				\nextgroupplot
				\addplot[color = Emerald] table[x=Time, y = y] from \phidata;
				\addlegendentry{$y$}
		\end{groupplot}
	\end{tikzpicture}
	\caption{$\phi$ och $\frac{a}{g}$ som funktioner av tiden konkatenerade från 7 loggar och kontrasterade mot den beräknade lutningen från GPS:en. \label{fig:phiresult}}
\end{sidewaysfigure}

Figur~\ref{fig:netresult} visar neuronnätets aktiveringar, $a$, på
lutningsgivarens råvärde och bussens acceleration,
tillsammans med den beräknade lutingen från GPS:en, $y$.

\begin{figure}
	\centering
	\begin{tikzpicture}
		\begin{axis}[
				no markers,
				legend style={at={(0.02,0.98)},anchor=north west},
				table/x expr=2 * 0.02 * 10 * \coordindex,
				xlabel = Tid, x unit = s,
				ylabel = Lutning, y unit = \%,
				]
				\addplot table[y=a] from {predict7.dat};
				\addlegendentry{$a$}
				\addplot table[y=y] from {predict7.dat};
				\addlegendentry{$y$}
		\end{axis}
	\end{tikzpicture}
	\caption{Neuronnätets beräknade lutning gentemot GPS:ens lutning. \label{fig:netresult}}
\end{figure}

I både figur~\ref{fig:phiresult} och figur~\ref{fig:netresult} innebär
positiv och negativ lutning att bussen kör i en uppförs- respektive nedförsbacke.
Enheten är procent, det vill säga,
höjdskillnaden i meter per 100 meters horisontell förflyttning.
