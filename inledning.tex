\section{Inledning}
\subsection{Teori \& bakgrund}
%bakgrund
% - buss
% - växellåda

Automatiska växellådor sitter idag i en bredd av olika fordon, däribland en
helt elektrisk buss under utveckling av Volvo.
För att bland annat köra på rätt växel, optimera motorns styrprogram och operera
olika bromsfunktioner %och vad mer/något mer?
installeras lutningsgivare som mäter lutningen fordonet framförs i.
Lutningsgivare är en fristående moduler som baseras på piezoelektriska
kristaller och kommunicerar med fordonets styrdator via Controller
Area Network-kommunikation (CAN-).
Dessa fungerar bra så länge som de är stilla, under rörelse (alltså vid
användning)  uppstår det mätbrus och andra signalstörningar.
Till den elektriska bussen ska en ny lutningsgivare utvecklas och utvärderas,
vilket innebär att en algoritm som kan felkompensera dessa störningar behöver
utvecklas.

%teori
% - piezoelekrtiskakristaller
% - algoritmen

Piezoelektriska material, har en kristallstruktur och, ger upphov till
elektriska laddningar på dess yta under yttre mekaniskt tryck, den så kallade
direkta piezoelektriska effekten.
Det mekaniska arbete som utförs omvandlas till elektricitet, det omvända gäller
också, elektricitet omvandlas till mekansikt arbete i den omvända
piezoelektriska effekten i vilken kristallen deformeras.
\autocite{electronicdesign2016}

\subsection{Syfte}
Syftet med  undersökningen var att ta fram en algoritm med neuronnät som
felkompenserar lutningsmätare.

\subsection{Frågeställningar}
Vi ska undersöka hur ett neuronnät ska vara konfigurerat för att kunna
felkompensera en lutningsmätare monterad i en lastbil; hur många inputs,
outputs, neuroner, och hidden layers som behövs.
Och hur nätverkets prestanda kan förbättras.
