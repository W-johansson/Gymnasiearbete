\section{Inledning}
\subsection{Teori \& bakgrund}
%bakgrund
% - buss
% - växellåda

Automatiska växellådor sitter idag i en bredd av olika fordon, däribland en
helt elektrisk buss under utveckling av Volvo.
För att bland annat köra på rätt växel, optimera motorns styrprogram och operera
olika bromsfunktioner %och vad mer/något mer?
installeras lutningsgivare som mäter lutningen fordonet framförs i.
Lutningsgivare är fristående moduler som baseras på piezoelektriska
kristaller och kommunicerar med fordonets styrdator via Controller
Area Network-kommunikation (CAN-).
Dessa fungerar bra så länge som de är stilla, när de accelererar (alltså vid
användning)  uppstår det mätbrus och andra signalstörningar.
Till den elektriska bussen ska en ny lutningsgivare utvecklas och utvärderas,
vilket innebär att en algoritm som kan felkompensera dessa störningar behöver
utvecklas.

%teori
% - piezoelekrtiskakristaller
% - algoritmen

Piezoelektriska material, har en kristallstruktur och, ger upphov till
elektriska laddningar på dess yta under yttre mekaniskt tryck, den så kallade
direkta piezoelektriska effekten.
Det mekaniska arbete som utförs omvandlas till elektricitet, det omvända gäller
också, elektricitet omvandlas till mekansikt arbete i den omvända
piezoelektriska effekten i vilken kristallen deformeras.
\autocite{electronicdesign2016}

\subsection{Syfte}
Syftet med undersökningen är att ta fram och utvärdera en algoritm med neuronnät som felkompenserar lutningsgivare i elbussväxellådor.

\subsection{Frågeställningar}
Vi vill undersöka\ldots
\begin{itemize}
	\item Varför blir råsignalen från lutningsgivaren opålitlig?
	\item Hur ska nätverket vara konfigurerat för att felkompensera; vilka inputs,
		outputs, antal neuroner och hidden layers?
	\item Hur väl fungerar det att felkompensera med neuronnät`
\end{itemize}
