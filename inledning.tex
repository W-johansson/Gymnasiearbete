\section{Inledning}
\subsection{Teori \& bakgrund}
% Bakgrund

Automatiska växellådor sitter idag i en mängd av olika fordon,
däribland en helt elektrisk buss under utveckling av Volvo.
För att bland annat köra på rätt växel, optimera motorns styrprogram
och operera olika bromsfunktioner % och vad mer/något mer?
mätes fordonets lutning.
För det ändamålet ska lutningsgivare användas.
Lutningsgivaren är en fristående modul som kommunicerar med styrdatorn via
Controller Area Network (CAN)-kommunikation.
Detta fungerar bra vid rörelse i konstant hastighet
men vid accelereration uppstår mätbrus och andra störningar.
Därför behövs en algoritm för att felkompensera dessa störningar.

%teori
% - piezoelekrtiskakristaller
% - algoritmen

Piezoelektriska material, har en kristallstruktur och, ger upphov till
elektriska laddningar på dess yta under yttre mekaniskt tryck, den så kallade
direkta piezoelektriska effekten.
Det mekaniska arbete som utförs omvandlas till elektricitet, det omvända gäller
också, elektricitet omvandlas till mekansikt arbete i den omvända
piezoelektriska effekten i vilken kristallen deformeras.
\autocite{electronicdesign2016}
Piezoelektriska lutningsgivare använder elektriska signaler som är
inducerade via den piezoelektriska effekten från en piezoelektrisk kropp
under gravitationskraften från en tyngd.
Vinkeln mellan gravitationskraften och
riktningen på den piezoelektriska kroppens vibration
fås av att man mäter magnituden av kraftenkomponenten i vibrationens riktning
och använder geometriska samband mellan dem.
\autocite{chiang00}

\subsection{Syfte}
Syftet med undersökningen är att ta fram och utvärdera en algoritm med neuronnät som felkompenserar lutningsgivare i elbussväxellådor.

\subsection{Frågeställningar}
Vi vill undersöka\ldots
\begin{itemize}
	\item Varför blir råsignalen från lutningsgivaren opålitlig?
	\item Hur ska nätverket vara konfigurerat för att felkompensera; vilka inputs,
		outputs, antal neuroner, hidden layers och feed-forward/cascade-forward?
	\item Hur väl fungerar det att felkompensera med neuronnät`
\end{itemize}
